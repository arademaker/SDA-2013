
\section{Conclusion}

% The new database system composed by Markdown files and periodic RDF
% releases can be easily used for accessing data in an efficient
% way. The figure below illustrates the process. CPDOC and their
% collaborators would be responsible for creating Markdown files that
% would be automatically analyzed by a script for generating the RDF
% file, which would be refined and improved. This refined RDF would be
% made available to the web as linked open data, which would allow the
% information to be accessed and improved by the community. On the other
% hand, the RDF would also be stored in a triple store that allows for
% queries using many different GUI configurations (for instance using
% Solr) to easy the access of data stored. This schema follows a modern,
% open and scalable way of sharing and improving the data stored in
% CPDOC.

In this paper we presented a new architecture for CPDOC archives
creation and maintenance. The architecture targeted is based on open
linked data concepts and open source methodologies and tools. We
believe that despite the fact that CPDOC users would need to be
trained to use the proposed tools such as text editors, version
control softwares and command line scripts; this architecture would
give more control and easiness for data maintenance. Moreover, the
architecture allows knowledge to be easily incorporated to collections
data without the dependency of database refactoring. This means that
CPDOC team will be much less dependent from FGV's Information
Management and Software Development Staff.

% NÃO SEI SE VALE A PENA FLOREAR A CONCLUSÃO. CORTE, SE ESTIVER
% DEMAIS!  Por fim, entendemos que as possibilidades abertas nesse
% cenário transcendem em muito os limites do mundo digital e
% repercutem diretamente na nossa própria história em construção, na
% maneira como queremos colaborar para a mediação do saber. O ato de
% conhecer não se reduz a uma apreensão inerte de dados, ou tão
% somente advém de apreciações meramente lógicas. As pessoas conjugam
% faculdades cognitivas e perceptivas quando interagem entre si e com
% os recursos que as cercam, participando da construção do
% conhecimento sobre a realidade. O ambiente digital e de rede traz
% infinitas oportunidades de compartilhamento dessa história em
% construção, com desdobramentos na cultura e no aprendizado. Mas para
% que esse ambiente seja efetivamente transformador, é fundamental o
% papel das políticas de difusão e disponibilização dos registros
% históricos, bem como a busca pela sua melhor forma de acesso,
% garantindo a diversidade da qual não podem prescindir.

% Finally, we believe that the possibilities offered by our proposed
% architecture far transcend the limits of the digital world and
% directly affect our own \textit{history in the making}, the way we
% collaborate for the mediation of knowledge. The act of knowing is
% not reduced to an inert seizure of data, nor it comes from
% assessments merely logical. People combine cognitive and perceptual
% faculties as they interact with the resources that surround them,
% participating in the construction of the knowledge over the
% reality. The digital environment and network brings endless
% opportunities for sharing this history in the making, with
% developments in culture and learning. But for this environment to be
% effectively transformer, is crucial the role of policy dissemination
% and availability of historical records, as well as the search for
% the right way to access it, ensuring the diversity which can not do
% without.

Many proposals of research concerning the use of lexical resources for
reasoning in Portuguese using the data available in CPDOC are being
carried out so as to improve the structure and quality of the DHBB
entries. Moreover, the automatic extension of the mapping proposed in
Section~\ref{sec:mapping} can also be defined following ideas of
\cite{onto-context}. Due the lack of space, we did not present in this
paper.

%%% Local Variables: 
%%% mode: latex
%%% TeX-master: "article"
%%% End: 
