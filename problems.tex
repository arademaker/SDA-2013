
\section{Current Status}\label{sec:problems}

In this section we summarize the main problems identified in CPDOC's
current infrastructure and daily working environment.

As described in Section~\ref{sec:cpdoc}, CPDOC's archives are
maintained by three different information systems based on traditional
relational data models. This infrastructure is hard to maintain,
improve and refine, and the information is not found or accessed by
standard search engines for two reasons mainly: (1) an entry page does
not exist until it is created dynamically by an specific query; (2)
users are required to login in order to make queries or access the
digital files. Service providers do not access data directly and
therefore cannot provide specialized services using it. Users
themselves are not able to expand the queries over the collections,
being limited to the available user interfaces in the website.
Thereupon, data of CPDOC's collections is currently limited to what is
called ``Deep Web''~\cite{bergman2001white}.

% The information systems do not provide an intuitive and effective
% interfaces to their users, making the access to information difficult
% and limited.

% The information systems were designed and developed in the 90's and
% since then has undergone several modifications. The changes were
% motivated by different reasons: from requirements of users to
% substitution of obsolete technologies (database systems or web
% development frameworks and programming languages). A careful
% analysis of CPDOC systems would consider them to be outdated
% especially if compared with the current available systems and
% technologies for data storage, indexing and long-term preservation.

The maintenance of current different information systems is very
problematic. It is expensive, time demanding and
ineffective. Improvements are hard to implement and therefore
innovative initiatives are usually postponed. A relational database
system is not easily modified, because relational data models must be
defined \emph{a priori}, i.e., before the data acquisition's
stage. Moreover, changes in the database usually require changes in
system interfaces and reports. The whole workflow is expensive, time
consuming and demands different professionals with different skills
from interface developers to database administrators.

Concerning terminology, CPDOC's collections do not follow any metadata
standards, which hinders considerably the interoperability with other
digital sources. Besides, the available queries usually face
idiosyncratic indexing problems with low rates of recall and
precision. These problems are basically linked to the \emph{ad hoc}
indexing strategy adopted earlier to define database tables and
fields.

% The maintenance of 3 different information systems is expensive,
% time demanding and ineffective. Improvements are hard to implement
% and therefore innovative initiatives are usually postponed. We
% believe this is a side effect of the adoption of a relational data
% model and the traditional information systems maintenance
% methodology.
% % Regarding the items above, it is perhaps worth justifying in
% % technical detail the last one.
% In a relational database system, it is well known that it is
% expensive and hard to frenquently change the data model. This is due
% to the assumption that a relational data model must be defined
% \emph{a priori}, i.e., before the beginning of data
% acquisition. Moreover, information systems for maintenance of
% relational databases are also difficult to evolve. Changes in the
% database require adaptations in the system interfaces and
% reports. This workflow is expensive, time consuming and demands
% several different professionals with different skills: interface
% developers, database administrators etc.

% The CPDOC researchers rely on the FGV's technical staff team from
% the IT department for any database modification or user interface
% improvement. Even the addition of a new field in a table requires a
% whole workflow comprising: (1) the specification of the requirement;
% (2) the costs and time estimation from the technical staff; (3) the
% appropriate authorization; (4) the allocation of a technical
% profissional or team for a period of time considering the other
% FGV's priorities and demands. This workflow usually takes weeks and
% even months.
%
% Whenever distributed and interoperable data maintenance is required,
% the relational data model fails severely. First, it is not easy to
% keep the model and data independent. Second, data exchange is
% hindered by then different storage strategies for datatypes and
% encodes of strings. Third, relational data models are commonly
% flooded with several auxiliary tables necessary to store non-trivial
% relations like N-N relations or generalizations. And finally, in
% general, few model constrains are maintained in databases, and the
% ones created are also not easily interoperable.

% Concerning data storage, it is also worth mentioning that Acessus
% and Oral Historic Interviews collections are not stored in a single
% place. The actual files (digitalized documents) of personal archives
% and the files of digitalized interviews are scattered in different
% file servers.

Finally, data storage is also an issue. Digitized Acessus's documents
and Oral History's interviews are not stored in a single place, but
scattered in different file servers. The CPDOC database only stores
the metadata and file paths to the file servers, making it very
difficult to ensure consistency between files, metadata information
and access control policies.

% Whereas people make great use of standard search engines to query
% and share knowledge, it becomes certainly one of the greatest
% challenges faced by CPDOC when it comes to open their collections.

%%% Local Variables: 
%%% mode: latex
%%% TeX-master: "article"
%%% End: 
