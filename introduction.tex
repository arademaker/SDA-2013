\section{Introduction}\label{sec:intro}

% 1. the use of open/linked data vs. relational data
% 2. embracing the open source development model for data maintainnability
% 3. CPDOC ownership versus curatorship of the datasets
%
% The Center for Research and Documentation of Contemporary History of
% Brazil (CPDOC) of Getulio Vargas Foundation (FGV) is an institution
% dedicated to study and preserve content from the Brazilian
% contemporary history.  CPDOC holds rich collections of personal
% archives, interviews and different audiovisual sources very
% important for understanding part of Brazilian memory. During the
% 70's and 80's, CPDOC team developed its own methodology for
% organizing and indexing documents, recognized as innovative by
% leading educational and research institutions. However, new
% technologies and trends for data sharing and interoperability of
% collections pose a challenge to CPDOC to remain innovative in its
% mission of efficiently dealing with historical data. It is time to
% adjust CPDOC methodology to new paradigms.

The Center for Teaching and Research in the Social Sciences and
Contemporary History of Brazil (CPDOC) was created in 1973 and became
an important historical research institute, housing a major collection
of personal archives, oral histories and audiovisual sources that
document the country memory. It is part of Getulio Vargas Foundation
(FGV), a prestigious Brazilian research and higher education
institution founded in 1944, considered by Foreign Policy Magazine to
be a top-5 ``policymaker think-tank'' worldwide \cite{think-tank}.

Thanks to the donation of the personal archives of President Getulio
Vargas and other brazilian figures who had been prominent from the
1930s onward, CPDOC started to develop its own methodology for
organizing and indexing documents, and by the end of the 1970s, it was
already recognized as a reference among research and historic
documentation centers. In 1975, the institute launched its Oral
History Program (PHO), which involved the execution and recording of
interviews with people who participated in major events in Brazilian
history. In 1984, CPDOC published the Brazilian
Historical-Biographical Dictionary (DHBB) \cite{dhbb}, a regularly
updated reference resource that documents the contemporary history of
the country. In the late 1990s, CPDOC was recognized as center of
excellence by the Support Program for Centers of Excellence (Pronex)
of the Brazilian Ministry of Science and Technology.

% % , after a dispute with several other institutions in the area of
% % humanities.
% In 2008, CPDOC has initiated a huge digitalization effort. 
% % In the last ten years it started offering graduate and undergraduate
% % courses in Social Sciences and History, receiving maximum score in
% % the latest assessment by the Brazilian Ministry of Education.
% Since it was launched, the DHBB has been an important source of
% information for researches, supporting the drafting of numerous theses
% and dissertations, as we can see in \cite{conniff,santos,paula}.

CPDOC is a vibrant and diverse intellectual community of scholars,
technicians and students, and has placed increasing emphasis on
applied research in recent years, working in collaborative projects
with other schools and institutes, aiming at extend the availability
and scope of the valuable records it holds. 

% However, trends for data sharing and interoperability of digital
% collections pose a challenge to the institute to remain innovative in
% its mission of efficiently dealing with historical data. It is time to
% adjust CPDOC methodology to new paradigms.

This year, when completes 40 years of existence, CPDOC will have the
support of the Brazilian Ministry of Culture (MinC), which is going to
provide a fund of R\$ 2.7 million to finance the project
``Dissemination and Preservation of Historical Documents'' wich has
the following main goals: (1) digitilizing a significative amount of
textual, iconographic and audiovisual documents; (2) updating the
dictionary DHBB; and (3) prospecting innovative technologies that
enable new uses for CPDOCs collections.

The advances in technology offer new modes of dealing with digital
contents; the number of approaches increases as new technological
tools are developed.  CPDOC is committed to offering swift access to
its archives and is working toward making all data available in a more
intelligent/semantic way, in the near future. In collaboration with
the FGV School of Applied Mathematics (EMAp), CPDOC is working on a
project that aims to enhance access to documents and historical
records by means of data-mining tools, semantic technologis and signal
processing. At the moment, two applications are being explored: (1)
face detection and identification in photographs, and (2) voice
recognition in the sound and audiovisual archives of oral history
interviews. Soon it will be easier to identify people in photographs,
and then search for them in the archives after they have been
identified. Additionally, voice recognition will help locate specific
words and phrases in audiovisual sources based on their alignment with
transcription -- a tool that is well-developed for English recordings
but not for Portuguese. Both of these processes involve machine
learning and natural language processing, since the computer must be
taught to recognize and identify faces and words.

% The progressive conversion of the analog to digital formats has
% reached a limit in which every product of human intellectual
% activity is born digital or soon will be available in some
% electronic format. The predominance of this modality brings changes
% not only in the way we elaborate and express knowledge, but also in
% how we acquire, store, share and use it.
   
Also, CPDOC wants that your data constitute a large knowledge base
accessible by the standards of semantic computing. Despite having
become a reference in the field of organization of collections, CPDOC
currently do not adopt any metadata standards nor any open data model
for them. Trends for data sharing and interoperability of digital
collections pose a challenge to the institution to remain innovative
in its mission of providing efficiently historical data. It is time to
adjust CPDOC's methodology to new paradigms.
   
% CPDOC is committed to offering swift access to its archival materials
% and is working toward making all data available in a more
% intelligent/semantic way, in the near future. This process includes
% engaging with the public and using innovative, new techniques in
% managing the archives. Currently these archives are not following any
% metadata standards nor any open data model that could facilitate
% interoperability between information systems.

% Metadata or meta information can be defined as "data that describes
% data '. Libraries, museums and documentation centers make use of
% metadata for purposes of cataloging and describing objects that
% integrate their collections, with a focus on access to
% information. Initiatives such as Dublin Core \cite{dc}, Encoded
% Archival Description (EAD) \cite{ead} and Text Encoding Initiative
% (TEI) \cite{tei} are some examples of metadata models developed for
% digital libraries and represent a major advance in the matter of
% interoperability and information exchange between
% repositories. Besides, some studies shows that the use of
% ontology-based metadata can be a promising way. In a case study,
% Weinstein \cite{Weinstein98ontology} carried out the conversion of a
% bibliographic catalog encoded in MARC format for a model of
% ontology, obtaining a knowledge base with descriptions much richer
% and expressive. This implies that the queries are mapped to the
% conceptual structure of the ontology, ie, the relationships embedded
% in it and not only by attributes values.
      
This article introduces a research project that reflects a change in
the way CPDOC wants to deal with archives maintenance and
difusion. The project is an ongoing initiative to build a model of
data organization and storage that ensures easy access,
interoperability and reuse by service providers. The project proposal
is inspired by: (1) Open Linked Data Initiative principles \cite{odi};
(2) distributed open source development model and tools for easy and
collaborative data maintenance; (3) a growing importance of data
curation concepts and practices for online digital archives management
and long-term preservation.
   
The project started with an initiative of creating a linked open data
version of CPDOC's archives and a prototype of a simple and intuitive
web interface for browsing and searching the archives. The uses of
Linked Open Data concept are conformed to the three laws first
published by David Eaves~\cite{3-law} and now widely accepted: (1) If
it can't be spidered or indexed, it doesn't exist; (2) If it isn't
available in open and machine readable format, it can't engage; and
(3) If a legal framework doesn't allow it to be repurposed, it doesn't
empower.
  
Our ideas are aligned with those proposed by the Semantic Web
community, it is an effort to build a global network of interconnected
open data. 

% This conception improves dramatically the possibility of discovering
% new knowledge as a straightforward consequence of adopting open
% standard formats with ensured accessibility.
Among the initiatives of this project we emphasize the construction of
a RDF~\cite{rdf-primer} data from data originally stored in a
relational database, the construction of an OWL~\cite{owl} ontology to
allow the proper represent the CPDOC domain.
% definition and interoperability of data in CPDOC. 
The proposal also aims of making the whole RDF data available for
download following DBPedia approach~\cite{dbpedia}. Brazil has a good
supply of public data available for free, but very few are in open
format, under the semantic web accepted standards. Examples in this
direction are the Governo Aberto SP~\cite{gasp}, the
LeXML~\cite{lexml} and the SNIIC project~\footnote{Sistema Nacional de
  Informações e Indicadores Sociais,
  \url{http://culturadigital.br/sniic/}.}.
   
This paper reflects a real effort grounded in research experience to
maintain CPDOC as a reference institution in the field of historic
preservation and documentation in Brazil.

% CPDOC has observed the evolution of our society by means of the
% irreversible process of globalization and informatization. This
% process has evolved together with new ways of learning and teaching,
% acquiring, storing and using data that allow people to use
% information as easily accessible knowledge. In this scenario, FGV, a
% youthful and nimble institution, remains aware of its precursor role
% of dealing with knowledge using new technologies.

% CPDOC is aware of that the process of digitalization and
% globalization of our society requires new methods of learning,
% teaching, acquiring, storing and using informations as so to
% transform it into actionable knowledge. In this scenario, FGV is
% uniquely positioned to lead in this new reality, being a youthful
% and nimble knowledge-based institution.

% Even though it is about the same universe of discourse -- the
% contemporary history of Brazil -- three different information
% systems were built to store the different types of documents.

% One of the greatest CPDOC's concerns is to make their archives
% widely visible online and interoperable with other digital
% archives. Currently these archives are not following any standard
% open data model, and queries to collections are limited to what
% systems in CPDOC offer, which is not much. The data is not
% accessible from outside and service providers cannot use CPDOC
% collections in their services, pontentially spreading the
% institution's audience.

% In the current state, CPDOC main concern is that its archives are
% not following any standard open data model. This scenario does not
% help to make the CPDOC's archives widely visible online and
% interoperable with other digital archives. Thus, queries to CPDOC's
% collections are limited to what CPDOC systems can offer and service
% providers could not use CPDOC collections in their services that
% could potentially reach a wide audience.

% We believe that CPDOC must review its methodology for digital data
% maintenance, storage, preservation and access policies. If the
% insitute starts focusing on the production and distribution of data
% and no longer in elaborated information systems and query
% interfaces, it will allow other forms of access, and therefore, new
% uses for these data, regardless of their interventions.

% There is much talk in Brazilian leading intellectual circles of
% connecting and relating to external centers of excellence, but if
% the mechanisms of interaction are kept restricted to the old models
% (traditional professorships and visiting lectureships for a year for
% well-established professors), the interaction with the more dynamic
% sectors such as entrepreneurs and incubators of start-up companies,
% will not develop at the pace it needs to.

% This article introduces a research project that reflects a change in
% the way CPDOC wants to deal with archives maintenance. The project
% is an ongoing initiative to build a model of data organization and
% storage that ensures easy access, interoperability and reuse by
% service providers. The project proposal is inspired by: (1) Open
% Data Initiative
% principles~\footnote{\url{http://www.opendatainitiative.org}}; (2)
% distributed open source development model and tools for easy data
% management and maintenance; (3) a growing importance of data
% curation concepts and practices for online digital archives
% management and long-term preservation.

%%% Local Variables: 
%%% mode: latex
%%% TeX-master: "article"
%%% End: 
